\section{Object flow}
\label{sec:core}

The object flow consist on computing the motion field for an object of interest through an image
sequence. The most usual approach to solve a problem like this is to implement some of the available
optical flow techniques through the complete sequence and perform the flow integration. 
However, this process results in high levels of motion drift \cite{c18}\cite{c19} and usually the motion of the interest
object is affected by a global regularization. In some extreme cases, the interest object motion
may be totally blurred and other techniques have to be incorporated. Moreover, the diversity
of natural video sequences makes difficult the choice of one technique over another, even when specialized
databases are at hand \cite{c17}, because currently no single method can achieve a strong 
performance in every of the available datasets. Most of these methods consist in the minimization 
of an energy function with two terms (As was previously mentioned in the Sec. \ref{sec:suppix}). The data
term is mostly shared between different approaches, but the prior or spatial term is different, and basically states 
under what conditions the optical flow smoothness should be maintained or not. In a global approach, however,
this is a difficult concept to define. Most of these smothness terms rely in appearance differences or gradients.
All these meaning that, unavoidably, some methods may be more reliable for some cases but weaker for others. 
It can be argued that this behaviour may be caused because most of the techniques do not count with a way to identify 
firmly where exactly this smoothness prior can be applied. 
   \begin{figure}[thpb]
      \centering
      \includegraphics[height=0.33\textheight]{images/object_flow.png}
      \caption{Object flow with the color code of \cite{c17} (Right) for one frame in the Puppy sequence (Left). Green Box: Current tracker state. Yellow: Segmentation contour.  }
      \label{of}
   \end{figure}
The main idea behid the object flow is that given the availability of several robust tracking techniques, and the proposed
segmentation method for video, the optical flow computation can be refined by computing it successively between pairs
of tracked windows. The basic proposal to perform this refinement consist on considering the segmentation limits  as reliable smoothness boundaries. 
This is, of course, under the assumption that the motion is indeed smooth within the object region. 
This is assumption is not far from reality in most scenes with an interest object. 
Of course, as the object tracker is included, is expected that the object flow should be more robust to rapid motion than the
optical flow. 
Thus, the full motion is split in two, the long range motion, given by the tracker window, and the precision part, given by the targeted optical flow. The Fig. \ref{of} shows 
the object flow for a frame in the Puppy sequence. Observe the motion vectors are computed only inside the object of interest, preserving a strong smoothing prior, but 
also allowing internal variations in the flow.

\subsection{Implementation details and results}

As a first approximation to the object flow, the Simple Flow technique \cite{c21} is taken as core base. This is because to its scalability 
to higher resolutions and because its specialization to the concept of object flow is only natural. This is because in the Simple Flow pipeline 
the smoothness localization can easily especified trough computation masks. More specifically, the initial computation mask is derived from 
the segmentation performed as prior step. The resulting flow is then filtered only inside mask limits to enhance precision and fastening the 
implementation. However, direct modifications in other optical flow methods can be further studied. For instance, in graph-cut based 
minimization approaches, the regularity constraints can be precisely targeted by disconecting foreground pixels from background ones.

   \begin{figure}[thpb]
      \centering
      \includegraphics[width=1.00\textwidth]{images/extrapolated.png}
      \caption{Extrapolation results from integrated flow in one frame of the Amelie Retro sequence. From Left to Right: Anotated object, Backward object flow, Backward optical flow, Forward object flow, Backward optical flow.}
      \label{sample}
   \end{figure}

To evaluate the performance of the object flow in comparison with optical flow techniques, we performed 
a number of experiments on several video sequences. We anotated an initial bounding box for the videos, 
and a segmentation contour of the interest object for every frame. The experiment measures the ability of the method to 
extrapolate an image from the initial frame and the integrated flow. For every pair of frames the PSNR between the anotated
current state of the object and the extrapolated images is computed. The Fig. \ref{sample} is a sample of the performed experiment, each column is an image generated from the given flow.

The Fig. \ref{of_res} shows PSNR graphics for 4 different sequences. For every pair of frames an image is extrapolated, and the PSNR is computed.
The measure is computed using Euler integration (Labeled as $forward$ in the figs.) of the used flow (object or optical flows), 
and using the integration method described in \cite{c20}, labeled as $backward$ in the figures.

   \begin{figure}[thpb]
      \centering
      \includegraphics[height=0.85\textheight]{images/ResultsOF.png}
      \caption{PSNR graphs for extrapolated images using Object flow and the Simple Optical Flow for 4 sequences. In descendent order: Puppy Seq.; Amelie Retro Seq.; Boy Seq.; Walking Seq.}
      \label{of_res}
   \end{figure}
