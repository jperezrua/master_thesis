%\section{Superpixel propagation for object segmentation in videos}
% \subsection{Background regions tracking for object segmentation}
\label{sec:segm}
Determining the spatial support of the object in a given frame benefits from the output of the tracker. Appearance cues alone, learned inside and outside the tracking window can result in a misleading modeling of the foreground and the background. In contrast, we propose to perform foreground-background segmentation by tracking background pixels surrounding the target, thanks to the tracker output. Thus, the pixels that are initially outside the tracker window, 
are followed through the sequence and as long as they enter the tracked region, they can be safely labeled as background. This idea can be observed in the Fig.  \ref{figurelabel_entering}, 
where the object window given by the tracker (green) loosely separates the foreground from the background. Points outside the tracker (blue) are labeled as background in previous frames, and as they enter the tracker window (red points), they can be used to improve the modeling  of the foreground and the background. The red window is used to save computational power by avoiding to track points that are too far from the interest object.

\ifcsdef{r@accv_format}{
   \begin{figure}[thpb]
      \centering
      \includegraphics[height=0.175\textheight]{../images/tracking_points.png}
      \caption{Example image of points entering a tracking region (green) due to object motion in a video sequence.}
      \label{figurelabel_entering}
   \end{figure}
\setlength{\belowcaptionskip}{-10pt}
}{
   \begin{figure}[thpb]
      \centering
      \includegraphics[width=0.5\textwidth]{../images/tracking_points.png}
      \caption{Example image of points entering a tracking region (green) due to object motion in a video sequence.}
      \label{figurelabel_entering}
   \end{figure}
\setlength{\belowcaptionskip}{-10pt}
}

Tracking pixels as independent points, however, carries several problems. First, to track all the background points means to compute the optical flow between the image pair, which can become very inneficient 
when taking into account large objects. Second, the point tracking techniques causes some drift in the trajectories, which can lead to wrong labelling as background. Finally, the background labelling could be 
more dense, and less complex if superpixels are tracked instead of pixels.  After several frames, 
the labeled superpixels will almost completely cover the unwanted areas in a dinamyc scene. We call this process background segments tracking.  Fig. \ref{figurelabel_spflow} shows this idea in a real scenario. From left to right, initially the superpixels with elements outside the bounding box are labeled as background (green), then, as the sequence runs, the labeled superpixels flow inside the window, propagating the background mask. 
The segmentation is then refined  applying the grab-cut method (\cite{c18}\cite{c15}) guided by the background labeling, which works a an initialization, replacing the usual user interaction.

\ifcsdef{r@accv_format}{
   \begin{figure}[thpb]
      \centering
      \includegraphics[width=0.95\textwidth]{../images/suppixflow2.png}
      \caption{Background segments automatic labeling and propagation, the flow goes from left to right. The superpixels that are outside the tracker window (yellow) are labelled as background (green) in the first frame. These labels are propagated 
when the sequence runs.}
      \label{figurelabel_spflow}
   \end{figure}
}{
   \begin{figure}[thpb]
      \centering
      \includegraphics[width=0.5\textwidth]{../images/suppixflow2.png}
      \caption{Background segments automatic labeling and propagation, the flow goes from left to right. The superpixels that are outside the tracker window (yellow) are labelled as background (green) in the first frame. These labels are propagated 
when the sequence runs.}
      \label{figurelabel_spflow}
   \end{figure}
}

%\subsection{Superpixel flow}
\label{sec:suppix}
%As a preprocessing step in the object flow pipeline, we propose a superpixel matching technique which assumes a flowlike behavior in the image 
%sequences (natural video), which can be used to track superpixels. 
%This matching, however, has to comply with a set of constraints. 
%Firstly, two correspondent superpixels should be similar in terms of some appearance
%feature, which most likely depends on the way the superpixelization was performed (color, texture,
%shape). Also, the superpixel flow  should maintain certain global regularity (at least for
%superpixels that belong to the same object). %In this sense, it seems

%If the size compactness of the superpixels is maintained,  it actually seems to 
%share some of the properties of the optical flow problem, with the difference that the
%smoothness is usually a very strong constraint for the last one. 
%The strength of this smoothness prior relies not only in the nature of the problem, but also
%because it gives better cues towards an easier-to-minimize global approach.

In order to perform the background segments tracking,  we propose a superpixel matching technique, 
which we call superpixel flow. The objective of the superpixel flow is to find the best match for every superpixel $p$ in the first 
frame with one ($p'$) in the next frame, while holding a global flow-like behaviour.

Thus, the superpixelization should maintain certain size homogeneity within a single frame. Some super
pixel techniques can cope with this requirement \cite{c9}\cite{c10}. For the experiments presented 
in this work, we prefer the SLIC method \cite{c9}, which gives
good results in terms of size homogeneity and compactness of the superpixelization. 

%\subsection{Energy Formulation}

Inspired by a large number of optical flow and stereo techniques \cite{c7}\cite{c12}\cite{c13}, 
the superpixel flow is modelled with a pairwise Markov Random Field. The matching is performed via maximum-a-posteriori (MAP) inference on the labeling $l$, which is equivalent to the minimization of an energy function of the form:

\begin{equation}
E(l) = \displaystyle \sum_{p \in \Gamma} D_p(l_p;I_0,I_1) +
\sum_{(p,q): q \in \mathcal{N}_r} S_{p,q}(l_p,l_q) .
\label{eq_energy}
\end{equation}

With $l$ the set of labels of the superpixels in $I_0$,
that match with those in $I_1$. $\mathcal{N}_r$ is a neighbourhood of radius $r$ of the superpixel $p$.
The terms $D$, and $S$ in (\ref{eq_energy}) stand for data term and spatial smoothness term. 
The first one determines how accurate is the labeling in terms
of consistency with the measured data (color, shape,etc.). In the classical optical flow equivalent of this equation,
the data term corresponds to the pixel brightness conservation\cite{c7}\cite{c5}. However, as superpixels are a set
of similar (or somehow homogeneous) pixels, an adequate appearance based feature is a low dimensional
color histogram (with N bins). So, here, $D$ is the Hellinger distance between the histograms:

\begin{equation}
D_p(l_p;I_0,I_1) = \sqrt{ 1 - \frac{1}{\sqrt{\bar{ \boldsymbol{h} }(p) \bar{ \boldsymbol{h} }(p')N^2} } \sum_{i}\sqrt{ \boldsymbol{h}_{i}(p) \boldsymbol{h}_{i}(p')} }.
\label{eq_Dp}
\end{equation}
Where $\textbf{h}(p)$ and $\textbf{h}(p')$ are the color histograms of the superpixel $p$ and its correspondent superpixel in the
second frame $I_1$. For the experiments we used {\it RGB} color histogram with N=3 bins per color.
Note that the low dimensional histogram gives certain robustness against noise,
and slowly changing colors between frames. 

On the other hand, the spatial term is a penalty function for spatial diference of the displacement vectors between neighboring superpixels, where a displacement vector has origin 
in the centroid of the superpixel of the first frame and
end in the centroid of the superpixel of the second frame.

\begin{equation}
S_{p,q}(l_p, l_q) = \lambda(p)
  \sqrt{\frac{|u_{p_c}-u_{q_c}|}{\|p_c-q_c\|}+ \frac{|v_{p_c}-v_{q_c}|}{\|p_c-q_c\|}},
\label{eq_Spq}
\end{equation}
 where, $ \lambda(p) = (1 + \rho(\boldsymbol{h}(p),\boldsymbol{h}(q)))^2 $.

The operator $\rho$ is the Hellinger distance as used in the
data term (\ref{eq_Dp}). The histogram distance is nonetheless computed between adjacent superpixels $p$ and $q$, 
which belong to the first image. The superpixels centroids are noted as $q_c$ and $p_c$, 
and $u_{*}$ and $v_{*}$ are the horizontal and vertical changes between centroids.
This term has a smoothing effect in superpixels that belong to the
same object. It has to be observed that when two close superpixels are different, thus, more probable to
belong to different objects within the image, the term $\lambda$ allows them to have
matches that do not hold the smoothness prior with the same strength. 

%\subsection{Energy Minimization}
\ifcsdef{r@accv_format}{
   \begin{figure}[thpb]
      \centering
      \includegraphics[width=0.935\textwidth]{../images/all_matches.png}
      \caption{The yellow lines show selected superpixel
		matching between pairs of images in several datasets.}
      \label{figurelabel_matchessnow}
   \end{figure}   
	\setlength{\belowcaptionskip}{-10pt}
}{
   \begin{figure}[thpb]
      \centering
      \includegraphics[width=0.5\textwidth]{../images/all_matches.png}
      \caption{The yellow lines show selected superpixel
		matching between pairs of images in several datasets.}
      \label{figurelabel_matchessnow}
   \end{figure}   
	\setlength{\belowcaptionskip}{-10pt}
}

The Quadratic Pseudo-Boolean Optimization (QPBO) \cite{c3}\cite{c4} is used to minimize the proposed energy function, 
by merging a set of candidate matches for every superpixel in the first frame. The candidate matches are generated by assuming 
a proximity prior. This means, every possible match should be inside a seach radius in the second frame. Fig. \ref{figurelabel_matchessnow} shows matching results for several datasets. Observe that the matches are correct even in difficult cases (bottom right).


\ifcsdef{r@accv_format}{
   \vspace*{-0.2cm}
   \begin{figure}[thpb]
      \centering
      \includegraphics[width=0.775\textwidth]{../images/Sequence2.png}
      \caption{Segmentation through the sequence “Walking
	       Couple” (Yellow contour) initialized in the man’s head. The yellow box correspond to the tracker output.
	        The labeled background superpixel are not shown for clarity.}
      \label{figurelabel_walking}
   \end{figure}
}{
   \begin{figure}[thpb]
      \centering
      \includegraphics[width=0.5\textwidth]{../images/Sequence.png}
      \caption{Segmentation through the sequence “Walking
	       Couple” (Yellow contour) initialized in the man’s head. The yellow box correspond to the tracker output.
	        The labeled background superpixel are not shown for clarity.}
      \label{figurelabel_walking}
   \end{figure}
}

Fig. \ref{figurelabel_walking} shows segmentation results for an image sequence where the interest object is the head of a person.
The head tracker and the superpixel flow provide information for better background-foreground separation. The method is tested in the Walking Couple sequence, by allowing only a small amount of iterations in the
grab-cut segmentation. Observe how the contour in the man's head is correctly delineated when
another person's head occludes part of it. In this case, the superpixels that belong to the woman’s face
were correctly propagated and thus, labeled as background, despite the similar (skin) color between foreground and background regions.

In order to understand the effect of including superpixel propagation in a video sequence for object
segmentation, some results are shown in the Fig. \ref{figurelabel_comp}. For these experiments only one iteration is
allowed in the graph-cut based methods. The top row frames (Fig. \ref{figurelabel_comp}) were initialized only with the tracker, 
and the bottom row was initialized with the superpixel tracking technique. 
Observe that in general, the contour delineated is usually better in terms of precision and
stability for the later one.
\vspace*{-0.5cm}
\ifcsdef{r@accv_format}{
   \begin{figure}[thpb]
      \centering
      \includegraphics[width=1.0\textwidth]{../images/compareSegm2.png}
      \caption{Face segmentation in the Snow Shoes sequence and T-shirt extraction from 
	      “Tennis” sequence in several frames. For each
	       group, the Top Row: One-iteration grab-cut initialized with tracker window;
	       and the Bottom Row: One-iteration grab-cut initialized with the background regions tracking.}
      \label{figurelabel_comp}
   \end{figure}
}{
   \begin{figure}[thpb]
      \centering
      \includegraphics[width=0.5\textwidth]{../images/Compare.png}
      \caption{Face segmentation in the “Amelia Retro” and the
	      “Snow shoes” sequences in three different frames. For each
	       group, the Top Row: One-iteration grab-cut initialized with tracker window;
	       and the Bottom Row: One-iteration grab-cut initialized with the background regions tracking.}
      \label{figurelabel_comp}
   \end{figure}
}
