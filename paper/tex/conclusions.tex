\section{Conclusions}

A framework to combine tracking and optical flow methods to improve 
object based dense motion description is presented. The pipeline is 
composed of three main steps, object tracking, segmentation and 
flow estimation. For the segmentation step a new promising video object 
segmentation algorithm was proposed, and, to the best of our knowledge, 
the introduced superpixel flow is the first energy based algorithm for superpixel matching.
For the last step, we presented a flow estimation method based on a modification of the simple-flow method to use 
the obtained segmentation mask. The experiments showed that this object based flow estimation improves the dense motion 
estimation for an object in comparisson to optical flow techniques.
Future work includes to explore the use of the object flow as feedback hint for tracking-by-detection methods. 
Furthermore, the use of other optical flow techniques as base of the object flow is a matter of great interest as more precise methods could be found.
Finally, several kind of applications of the object flow can be more deeply approached. For instance, 
in the structure-from-motion pipeline, video editing and video inpainting, among others.



%A method for superpixel matching in a flow-like
%behavior had been presented. We may call it superpixel propagation, or super-pixel flow. This
%technique shows robustness in the labeling of the
%matches, and seems to be able to improve the results
%for object segmentation in video sequences when
%using the Grab-cut algorithm. Another motivation for
%this technique, however may be as initialization of
%optical flow techniques, to obtain more robust results
%due to a good initial guess for energy minimization
%methods. This has to be looked at more deeply in future work.