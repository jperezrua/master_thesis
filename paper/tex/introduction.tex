\section{Introduction}
\label{sec:introduction}

Object tracking and optical flow are two of the main components in the
computer vision toolbox, and have been focus of great research efforts, 
leading to significant progress in the last years \cite{c16}\cite{c17}. 
The object tracking problem consist on estimating the 
position of the target in future frames, given an initialization. In the
other hand, the optical flow between a pair of frames consist on finding a motion vector 
for each pixel of interest in the initial image. Even though for several
applications a full motion-field is needed, other applications like
human-computer interaction, object editing in video or structure-from-motion,
may only focus on an interest object and, thus, only motion vectors within its 
space may be of interest. 
In such scenarios combining optical flow and object tracking in a unified 
framework would become useful and the precision of the object motion description 
could be enhanced. For instance, even with modern optical flow approaches, 
the long term motion problem remains a challenge. However, the problem is more 
bearable for object tracking techniques. In contrast, object trackers are more global 
in motion description, and its information can be completed by optical flow sub-pixel 
precision. Moreover,  even when object
trackers and optical flow could give good hints for object segmentation in video, 
these elements are not deeply studied in the literature as a unified problem.
We introduce the object flow problem as the computation of dense motion 
flow fields of the set of pixels that belong to an interest object. In other words, 
the object flow by definition induces the segmentation of the target and its motion field.

We can define more precisely the object flow by starting with an image sequence and an initial 
position of the interest object in the first frame of this sequence, and letting $\mathcal{R}$ be the region corresponding to the support of the object in 2D, such that 
$\mathcal{R} \subset \Omega$. If $\Omega$ is the set of all the possible grid positions, 
the object flow problem consist in finding the displacement vector $d_{0,t}(x)$ from the image $I_{0}$ to $I_{t}$, $\forall x \in \mathcal{R}$.

A straightforward solution to this problem would be to compute the optical flow motion field, and apply 
a segmentation mask to recover the desired motion vectors. Nevertheless, this approach carries several 
problems. For example, a globally computed optical flow method can affect small objects motion, because of 
the common use of heavy regularization priors. Moreover, even if the segmentation mask is extracted from a 
tracker position by a graph-cut based method, is likely that this mask is not going to be well suited for the 
interest object and some extra user interaction would be needed to refine this process. We propose an approach 
to reduce these problems.

%Among the state of the art segmentation methods for objects in video
%sequences, point trajectories based ones stand for its performance and 
%reliability, even when only sparse trajectories are known because of
%computational reasons. In the other hand, for the problem of extracting out a 
%preselected object in still frames, max-flow min-cut based approaches 
%have demostrated to be a powerful tool. We propose to mix these two 
%ideas together with the tracking of backgro und regions via the novel concept of 
%superpixel flow for reliable object segmentation through video. 
%We show how this extra information can be used to complement the
%graph-cuts based techniques for an efficient foregorund-background segmentation.

The present paper is organized as follows. We describe our pipeline for object flow, including the novel concept of superpixel flow
in Sec. \ref{sec:desc} and its use in object segmentation in videos. In following
sections some results showing how 
the object flow overpass state of the art optical flow methods for object motion flows 
estimation are discussed. Finally, some insights and conclusions are given.

