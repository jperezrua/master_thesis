\section{INTRODUCTION}
\label{sec:introduction}
Superpixels and over segmentation techniques
became a widely used pre-processing stage for a
large number of machine vision applications, after the
original concept was introduced \cite{c1}. Superpixels are
traditionally used as performance booster for several
other techniques. However, it is still mostly related to
single frame processing \cite{c1}\cite{c10}\cite{c11}. In the search for
consistency in superpixel labeling through video,
some authors have proposed different techniques,
which go from simple extension to supervoxels\cite{c9}\cite{c11},
to more complicated approaches \cite{c8}. These
approaches, nonetheless, usually require a global
processing and knowledge of all (or several of) the
video frames beforehand. One of the contributions of this work is a superpixel
matching technique which assumes a flowlike
behavior in the image sequences (natural video), and
propose an application for improving object segmentation in videos.

For some kind of video sequences and computer vision
applications the superpixel labelling usually loses coherency between frames, 
even when computed as supervoxels. 
This means superpixel matching techniques offer some interest and can still be
explored. Some work have been done towards a
superpixel based image comparison using the Earth
Mover's Distance, by taking super pixels as bins of a
global histogram \cite{c2}. The label propagation or superpixel flow can be
achieved with this technique by selecting the superpixel in the second frame that maximize the flow
from each superpixel in the first frame.
However, we look at the problem of superpixel
propagation between a pair of frames, in terms of the
indexes of the already computed superpixelization in
both of them. \\
By taking into account superpixels computed separately in images we can open the advantages of
streaming like approaches, so the video process can be performed by processing only two frames at a time, 
saving memory and moving towards a more time friendly approach. This propagation means finding
the superpixel labels in the second frame that
correspond to a given label in the first frame. This
matching, however, has to comply with a set of
constraints. Firstly, two correspondent superpixels
should be similar in terms of some appearance
feature, which most likely depends on the way the
superpixelization was performed (color, texture,
shape). Also, the superpixel propagation should
maintain certain global homogeneity (at least for
superpixels that belong to the same object). This is
because the superpixel propagation or matching is
not completely a one-to-one combinatorial problem,
but more a “superpixel flow”. In this sense, it seems
natural that the problem of superpixel propagation
could be solved with a discrete energy minimization
procedure. If the size compactness of the superpixels is maintained, 
it actually seems to share some of the properties of the
optical flow problem, with the difference that the
smoothness is usually a very strong constraint for the
last one. The strength of this smoothness prior relies
not only in the nature of the problem, but also
because it gives better cues towards an easy-to-
minimize global approach.