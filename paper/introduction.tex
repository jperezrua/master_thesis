\section{Introduction}
\label{sec:introduction}

Object tracking and optical flow are two of the main components in the
Computer Vision toolbox, and have been focus of great research efforts, 
leading to significant progress in the last years \cite{c16}\cite{c17}. 
The object tracking problem consist on estimating the 
position of the target in future frames, given an initialization. In the
other hand, the optical flow between a pair of frames consist on finding a motion vector 
for each pixel of interest in the initial image. Even though for several
applications a full motion-field is needed, other applications like
human-computer interaction, object editing in video or structure-from-motion,
may only focus on an interest object and, thus, only motion vectors within its 
space may be of interest. 
In such scenarions combining optical flow and object tracking in a unified 
framework would become useful and the precision of the object motion description 
could be enhanced. For instance, even with modern optical flow approaches, 
the long motion problem remains a challenge. However, the problem is more 
bearable for object tracking techniques. Moreover, even when object
trackers and optical flow give good hints for object segmentation in video, 
these elements are not deeply studied in the literature as a unified problem.
We introduce the object flow problem as the computation of dense motion 
flow fields of the set of pixels that belong to an interest object. In other words, 
the object flow by definition copes with segmentation of the target.

Among the state of the art segmentation methods for objects in video
sequences, point trajectories based ones stand for its performance and 
reliability, even when only sparse trajectories are known because of
computational reasons. In the other hand, for the problem of extracting out a 
preselected object in still frames, max-flow min-cut based approaches 
have demostrated to be a powerful tool. We propose to mix these two 
ideas together with the tracking of backgorund regions via the novel concept of 
superpixel flow for reliable object segmentation through video. 
We show how this extra information can be used to complement the
graph-cuts based techniques for an efficient foregorund-background segmentation.

The present paper is organized as follows. We introduce the concept of superpixel flow
in Sec. \ref{sec:suppix}. Then, a method for object segmentation in video which uses object tracker 
information and the background segments tracking is presented in Sec. \ref{sec:segm}. In following
sections the object flow basic approach is explained. Finally, results showing how 
the object flow overpass state of the art optical flow methods for object motion flows 
computation.

