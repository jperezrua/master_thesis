%%%%%%%%%%%%%%%%%%%%%%%%%%%%%%%%%%%%%%%%%%%%%%%%%%%%%%%%%%%%%%%%%%%%%%%%%%%
% Juan Manuel Perez Rua
%%%%%%%%%%%%%%%%%%%%%%%%%%%%%%%%%%%%%%%%%%%%%%%%%%%%%%%%%%%%%%%%%%%%%%%%%%%

%\section{Superpixel flow} \label{sec:suppix}
\chapter{Superpixel flow} \label{chap:suppix}

During the initial stages of the object flow pipeline definition, the use of superpixel 
methods was considered. However, as the object flow refers mainly to the problem of 
estimating motion for objects in video, the superpixel method studied had to be able to 
maintain consistency between frames. Thus, usual approaches for time-consistent superpixels 
were studied. These methods carried some disadvantages, as explained later, and a novel 
idea had to be introduced in order to develop the concept of the object flow as it 
was being thought. This novel idea is the problem of matching precomputed superpixel 
between frames, by maintaining the time-smoothness which is natural of videos. 
This is, to incorporate the idea of flowing pixels to the superpixel level.

\section{Problem definition}
Superpixels and over segmentation techniques became a widely used pre-processing 
stage for a large number of machine vision applications, after the
original concept was introduced \cite{c1}. Superpixels are traditionally used as 
performance booster for several other techniques. However, it is still mostly related to
single frame processing \cite{c1}\cite{c10}\cite{c11}. In the search for
consistency in superpixel labelling through video, some authors have proposed different 
techniques, which go from simple extension to supervoxels\cite{c9}\cite{c11},
to more complicated approaches \cite{c8}. These approaches, nonetheless, usually require a 
global processing and knowledge of all (or several of) the video frames beforehand, which would 
affect the desired online characteristic of the target application, in this case, of the object flow.

Thus, as a prerequisite concept in the object flow pipeline, we propose a superpixel matching technique which assumes a flow-like behaviour in the image 
sequences (natural video), which can be used to track superpixels. 
Some previous work have been done towards a
superpixel based image comparison using the Earth Mover's Distance, by taking superpixels 
as bins of a global histogram \cite{c2}. The label propagation or superpixel flow can be
achieved with this technique as a by-product, by selecting the superpixel in the second frame that 
maximize the EMD flow from each superpixel in the first frame.
By taking into account superpixels computed separately in images, so the video process can be 
performed with only two frames at a time, we move towards a more time efficient approach. 
This matching, however, has to comply with a set of constraints. 
Firstly, two correspondent superpixels should be similar in terms of some appearance
feature, which most likely depends on the way the superpixelization was performed (color, texture,
shape). Also, the superpixel flow  should maintain certain global regularity (at least for
superpixels that belong to the same object). In this sense, it seems
natural that the problem of superpixel flow could be solved with a discrete energy minimization
procedure. 
If the size compactness of the superpixels is maintained,  it actually seems to 
share some of the properties of the optical flow problem, with the difference that the
smoothness is usually a very strong constraint for the last one. 
The strength of this smoothness prior relies not only in the nature of the problem, but also
because it gives better cues towards an easier-to-minimize global approach.

The objective of the superpixel flow is therefore to find the best labelling $l$ for every superpixel $p$ (with $l_p \in {0,1,...M-1}$, between a pair of frames ($I_{0}$,$I_{1}$), but holding a flow-like behaviour. $M$ is the number of superpixels in the second frame.

Thus, the superpixelization should maintain certain size homogeneity within a single frame. Some super
pixel techniques can cope with this requirement \cite{c9}\cite{c10}. For the experiments presented 
in this work, the SLIC method \cite{c9} is preferred, because it usually gives
good results in terms of homogeneity of the superpixelization across the sequence. 
The proposed steps to solve the propagation problem assume this requirement is hold. 
For other kind of the techniques, other approaches should be followed.

%\addtolength{\textheight}{-3cm}   % This command serves to balance the column lengths
%%%%%%%%%%%%%%%%%%%%%%%%%%%%%%%%%%%%%%%%%%%%%%%%%%%%%%%%%%%%%%%%%%%%%%%%%%%%%%%%

\section{Energy Formulation}

Inspired by a large number of optical flow and stereo techniques \cite{c7}\cite{c12}\cite{c13}, 
the superpixel flow can be modelled with a pairwise Markov Random Field in the super pixel space $\Gamma$. If
the matching is performed with MAP inference, its posterior probability is: 

\begin{equation}
P(l|I_0,I_1) = \displaystyle \prod_{p \in \Gamma} \mathrm{e}^{-D_p(l_p;I_0,I_1)} 
\prod_{(p,q): q \in \mathcal{N}_r} \mathrm{e}^{-S_{p,q}(l_p;l_q)} ,
\label{eq_prob}
\end{equation}

With $l$ the set of labels of the super pixels in $I_0$,
that match with those in $I_1$.
$ \mathcal{N}_r $ is a neighbourhood of the
superpixel $p$ containing all the superpixels which are inside a circle of radius $r$ with center in $p_c$. This neighbourhood defines its adjacency, and its not of a constant size, like in the pixel 
equivalent of these kind of equations, where a 4 or 8 neighbourhood is defined.  
Given this posterior probability, the equivalent energy function can be directly obtained
by extracting the negative logarithm of the posterior,

\begin{equation}
E(l) = \displaystyle \sum_{p \in \Gamma} D_p(l_p;I_0,I_1) +
\sum_{(p,q): q \in \mathcal{N}_r} S_{p,q}(l_p,l_q) .
\label{eq_energy}
\end{equation}

The terms $D$, and $S$ in (\ref{eq_energy}) stand for data term and spatial smoothness terms as they
are popularly known in the MRF literature. The first one determines how accurate is the labelling in terms
of consistency of the measured data (color, shape, etc.). In the classical optical flow formulation of this equation,
the data term corresponds to the pixel brightness conservation\cite{c7}\cite{c5}. However, as superpixels are a set
of similar (or somehow homogeneous) pixels, an adequate appearance based feature can be a low dimensional
color histogram with $N$ bins. So $D$ can be written more precisely as the Hellinger distance between the histograms:

\begin{equation}
D_p(l_p;I_0,I_1) = \sqrt{ 1 - \frac{1}{\sqrt{\bar{h}(p)\bar{h}(p')N^2} } \sum_{i}\sqrt{h_{i}(p)h_{i}(p')} }
\label{eq_Dp}
\end{equation}

Where $h(p)$ and $h(p')$ are the histograms of the superpixel $p$ and its correspondent superpixel in the
second frame $I_1$. 
Note that the low dimensional histogram ($N=2, N=3$) gives certain robustness against noise,
and slowly changing colors between frames. 

On the other hand, the spatial term is a penalty function for horizontal
and vertical changes of the vectors that have origin in the centroid of the superpixel of the first frame and
end in the centroid of the superpixel of the second frame.

\begin{equation}
S_{p,q}(l_p, l_q) = \lambda(p)
  \sqrt{\frac{|u_{p_c}-u_{q_c}|}{\|p_c-q_c\|}+ \frac{|v_{p_c}-v_{q_c}|}{\|p_c-q_c\|}}
\label{eq_Spq}
\end{equation}
\begin{center}
 where, $ \lambda(p) = (1 + \rho(h(p),h(q)))^2 $ \\
\end{center}

In (\ref{eq_Spq}) the operator $\rho$ is the Hellinger distance as used in the
data term (\ref{eq_Dp}). The histogram distance is nonetheless computed between superpixels $p$ and $q$, 
which belong to the same neighbourhood. The superpixels centroids are noted as $q_c$ and $p_c$, 
and $u$ and $v$ are the horizontal and vertical changes between centroids.
This term is usual in the MRF formulation and has a smoothing effect in superpixels that belong to the
same object. It has to be observed that when two close superpixels are different, thus, more probable to
belong to different objects within the image, the term $\lambda$ allows them to have
matches that do not hold the smoothness prior with the same strength. 
It has to be noted that the proposed energy function is highly non-convex.

\section{Energy Minimization}

A fair amount of work has been dedicated to discrete optimization techniques in computer vision,
leading to well-defined and widely tested approaches to solve pairwise MRF\cite{c3}\cite{c4}.
However, some of the approaches restrict the construction of the spatial term, and/or enforce
limitations in the number of labels \cite{c3}.
Because of the high amount of possible labels for  each superpixel in the proposed approach, the use of the
Fusion Moves \cite{c7} technique seems to be well suited.
This algorithm employs the Quadratic Pseudo-Boolean Optimization (QPBO), to combine
incremental sets of proposal labellings, resulting in a semi-globally-optimal solution \cite{c4}.
Thus, the minimization starts by proposing a set of possible solutions, and iteratively merges them with
the QPBO technique. 

The candidate solutions depend on the problem to be solved. 
For example, in stereo superpixel matching, some assumptions related to the cameras 
layout can be made to generate solutions. In a more generic sense, other assumptions can be made towards 
candidate generation. 
The Quadratic Pseudo-Boolean Optimization (QPBO) \cite{c3}\cite{c4} is used to minimize the proposed energy function, 
by merging a set of candidate matches for every superpixel in the first frame.
For instance, for a given superpixel in the initial frame, the corresponding 
matching would be the most similar one in terms of color, shape, or the spatial distance. More candidate solutions can be added by defining a
neighbourhood in the second frame and select random pairs from every neighbourhood of every superpixel
in the first frame. This is suitable for problems where the images are extracted from the same video
sequence. This algorithmic process is presented more precisely in the Algorithm \ref{algo2}, where it 
can be seen that for every iteration, a single binary QBPO is optimized, and successively 
merged onto a final solution ($winner$). These solutions are extracted by selecting 
radial neighbourhood superpixels ($radius\_r$) in the frame $A$, to match with 
superpixels in a radial neighbourhood ($radius\_p$) in the frame $B$.

\begin{algorithm}[thpb]
\caption{Superpixel flow minimization algorithm}
\label{algo2}
\begin{algorithmic}
\REQUIRE list: $superpixelsA, superpixelsB$, float: $radius\_r, radius\_p$

graph: $graph$
vector: $option, winner$

\COMMENT{ Initialize winner labels }
\STATE $winner[p] = q | argmin( hellingerDistance(p, q) )$
	
\COMMENT{ Build unary terms }
\FORALL{$n iterations$}
	\STATE $graph.restart()$
	
	\FORALL{$p \in superpixelsA$}
		\FORALL{$q \in superpixelsB$}
	
    		\STATE $distance = hellingerDistance(p, q)$
    		\STATE $graph.addUnaryTerm(p, q, distance)$

		\ENDFOR
	\ENDFOR

	\FORALL{$p \in superpixelsA$}
		\STATE $p_c = centroid(p)$
		\FORALL{$q \in superpixelsA | distance(p_c,q_c)<radius\_r$}

		\STATE $option[q] = selectRandomSuppixInB( q_c, radius\_p )$
		
		\STATE $d00 = robustDistance(winner[p], winner[q])$
		\STATE $d01 = robustDistance(winner[p], option[q])$
		\STATE $d10 = robustDistance(option[p], winner[q])$
		\STATE $d11 = robustDistance(option[p], option[q])$
								
		\STATE $graph.AddPairWiseTerm(p,q,d00,d0,d10,d11)$
		\STATE $graph.solveQPBO()$
		\ENDFOR
	\ENDFOR
	
	\COMMENT{ Update winner labels }
	\FORALL{$p \in superpixelsA$}
		\STATE $winner[p] = graph.getLabel(p)$
	\ENDFOR
\ENDFOR

\RETURN $winner$
\end{algorithmic}
\end{algorithm}

In this case, the proposed approach is, to enforce a proximity prior and to extract possible matches from 
a circular neighbourhood in the subsequent frame centred on the current superpixel centroid. The radius of this neighbourhood is not necessarily 
of the same value as the radius $r$, and it accounts for the maximum possible matching distance between superpixels. Thus, this maximum search distance becomes a parameter to control the maximum 
amount of displacement of the superpixels. There is a direct relation between the value of this 
parameter and the processing time.
To speed-up the minimization procedure, the QBPO properties can be exploited. For instance, the fusion of the
proposed solutions is always guaranteed to be of lowest or equal energy than the two proposals. Thus, it does not matter the order in which the solutions are merged. 
One could, then, split the fusion procedure in several cores and build a hierarchical chain as proposals are subsequently fused. The idea of this implementation is compatible with 
the modern tendency to move the algorithms to the GPU level.

\section{Matching Results}

   \begin{figure}[thpb]
      \centering
      \includegraphics[width=1.00\textwidth]{../images/matches.png}
      \caption{The yellow lines show selected superpixel
		matching between pairs of consecutive frames in a video
		with the proposed method. The video frames go from right
		to left. The images are a close-up of the actual frame, to 
		appreciate the details.}
      \label{figurelabel_matches}
   \end{figure}
	
The Fig. \ref{figurelabel_matches} shows some examples of superpixel matching between subsequent frames with the presented method. 
It can be seen that the matching performs well even in difficult cases, like the hands in the first column ({\it yunakim long2} sequence). It has to be noted
as well that even in superpixels where there is a lack of texture, there is correct matching. This seems to be
the effect of enforcing the regularization between superpixels that are close, but are also similar to
each other. Moreover, unlike most of the optical flow methods, superpixel flow extends 
 naturally for more distant frames. 
The Fig. \ref{figurelabel_matchessnow} shows
 results for large separations between frames, without tweaking or adjusting any parameters. 
For this case, however, the matches in the texture-less part of the scene
 are mostly invalids. Though this is expected because of the aperture problem and
 heavy occlusions, thus, the matching in this region lacks of meaning.

   \begin{figure}[thpb]
      \centering
      \includegraphics[width=0.7\textwidth]{../images/matches_snowshoes2.png}
      \caption{The coloured lines show selected superpixel
		matchings between a pair of distant frames in the Snow Shoes sequence.}
      \label{figurelabel_matchessnow}
   \end{figure}   
	\setlength{\belowcaptionskip}{-10pt}