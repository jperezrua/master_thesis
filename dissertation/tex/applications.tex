\chapter{Applications} \label{chap:apps}

A couple of applications for the object flow are presented. Firstly, 
in complex video edition and augmented reality, and secondly as part 
of the object-based structure-from-motion. Only visual results are provided 
in both cases.

\section{Video edition}

The video edition that is approached here is the insertion or modification of 
an element in a sequence. For instance, during post-production it is possible that 
an element of a sequence presents inadequate content for the expected audience, and usually 
removing these elements require a hard manual art-edition work. The object flow 
can be exploited in order to reduce drastically the amount of manual work that has to be done. 

The proposed approach is to edit manually the initial frame. The elements of this edition 
can be propagated  through a number of subsequent frames with the object flow. Moreover, 
for complex scenes, with heavy occlusions and sudden illumination changes, several instances 
of the object flow can be used, to take into account the difficult cases. Thus, the overall edition 
work is greatly reduced.

   \begin{figure}[tpbh]
      \centering
      \includegraphics[width=0.95\textwidth]{../images/videoedition.png}
      \caption{  Selected frames in the test Dmitry test sequence for video edition. Top: Inserted logo. Bottom: original frames}
      \label{pt_seg}
   \end{figure}

\section{Structure from motion}

The idea of the structure from motion problem ({\it SfM}) is to 
recover the shape of objects or scenes from a sequence of images obtained from a camera that follows certain motion. Usually, it is assumed that 
the scene contains rigid objects undergoing an 
Euclidean motion.

The most common way to solve the {\it SfM} is to 
find a set of correspondences between the acquired 
frames by using a feature matching method, and use the epipolar geometry constraint for two images 
($x_1^T E x_2 = 0$). Usually, the {\it 8 points algorithm} is used to obtain a relative rotation 
and translation of the camera by factorizing the 
essential matrix ($E$). However, given the simplicity of the approach, the algorithm is very sensitive to noise and several other approaches have 
been proposed. One of the major weakness of such 
methods is still the feature matching (w.r.t precision and sparsity). 

When matching is replaced by optical flow, the 
scene structure and the camera motion are tied 
together by the equation:

\begin{equation}
\textbf{u}(x) = A(x)\textbf{v}/Z + B(x)\omega   
\label{eq_dfc}
\end{equation}
\begin{center}
 where, $ A =  \left[ {\begin{array}{ccc} 1 & 0 & -x\\ 0 & 1 & -y \end{array} } \right]$ \\
 and, $ B =  \left[ {\begin{array}{ccc} -xy & (1+x^2) & -y\\ -(1+y^2) & xy & x \end{array} } \right]$
\end{center}

The optical flow, however, has been traditionally 
inaccurate for long term point tracking, and due to the fact that an optimal structure in the least square sense can 
be obtained from camera velocities (which are more accurately extracted from the flow), the research was 
mainly focused on capturing camera ego-motion. 

As it has been demonstrated that the object flow 
increases accuracy of the point trajectories estimation, it can be connected to a {\it SfM} 
pipeline as point matching method. Some results can be appreciated in Fig. \ref{sfm1}. These results are generated 
by using the Changchang Wu's public tools based on his work on Linear-Time Incremental Structure From Motion, and the Furukawa's 
Clustering Views for Multi-view Stereo for merging the computed structures.

   \begin{figure}[t]
      \centering
      \includegraphics[width=0.95\textwidth]{../images/SFM.png}
      \caption{ SFM based on the object flow matches in the Crazyhorse dataset. Top: views of the computed structure. Bottom: Used frames. }
      \label{sfm1}
   \end{figure}

   \begin{figure}[t]
      \centering
      \includegraphics[width=0.95\textwidth]{../images/SFM2.png}
      \caption{ SFM based on the object flow matches in the Puppy dataset. Top: views of the computed structure. Bottom: Used frames. }
      \label{sfm1}
   \end{figure}